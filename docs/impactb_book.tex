\documentclass[]{book}
\usepackage{lmodern}
\usepackage{amssymb,amsmath}
\usepackage{ifxetex,ifluatex}
\usepackage{fixltx2e} % provides \textsubscript
\ifnum 0\ifxetex 1\fi\ifluatex 1\fi=0 % if pdftex
  \usepackage[T1]{fontenc}
  \usepackage[utf8]{inputenc}
\else % if luatex or xelatex
  \ifxetex
    \usepackage{mathspec}
  \else
    \usepackage{fontspec}
  \fi
  \defaultfontfeatures{Ligatures=TeX,Scale=MatchLowercase}
\fi
% use upquote if available, for straight quotes in verbatim environments
\IfFileExists{upquote.sty}{\usepackage{upquote}}{}
% use microtype if available
\IfFileExists{microtype.sty}{%
\usepackage{microtype}
\UseMicrotypeSet[protrusion]{basicmath} % disable protrusion for tt fonts
}{}
\usepackage{hyperref}
\hypersetup{unicode=true,
            pdftitle={IMPaC-TB: Integrated analysis and dynamical systems modeling of experimental TB immunology data},
            pdfauthor={Brooke Anderson PhD, Mike Lyons PhD, Amy Fox, Burton Karger},
            pdfborder={0 0 0},
            breaklinks=true}
\urlstyle{same}  % don't use monospace font for urls
\usepackage{natbib}
\bibliographystyle{apalike}
\usepackage{longtable,booktabs}
\usepackage{graphicx,grffile}
\makeatletter
\def\maxwidth{\ifdim\Gin@nat@width>\linewidth\linewidth\else\Gin@nat@width\fi}
\def\maxheight{\ifdim\Gin@nat@height>\textheight\textheight\else\Gin@nat@height\fi}
\makeatother
% Scale images if necessary, so that they will not overflow the page
% margins by default, and it is still possible to overwrite the defaults
% using explicit options in \includegraphics[width, height, ...]{}
\setkeys{Gin}{width=\maxwidth,height=\maxheight,keepaspectratio}
\IfFileExists{parskip.sty}{%
\usepackage{parskip}
}{% else
\setlength{\parindent}{0pt}
\setlength{\parskip}{6pt plus 2pt minus 1pt}
}
\setlength{\emergencystretch}{3em}  % prevent overfull lines
\providecommand{\tightlist}{%
  \setlength{\itemsep}{0pt}\setlength{\parskip}{0pt}}
\setcounter{secnumdepth}{5}
% Redefines (sub)paragraphs to behave more like sections
\ifx\paragraph\undefined\else
\let\oldparagraph\paragraph
\renewcommand{\paragraph}[1]{\oldparagraph{#1}\mbox{}}
\fi
\ifx\subparagraph\undefined\else
\let\oldsubparagraph\subparagraph
\renewcommand{\subparagraph}[1]{\oldsubparagraph{#1}\mbox{}}
\fi

%%% Use protect on footnotes to avoid problems with footnotes in titles
\let\rmarkdownfootnote\footnote%
\def\footnote{\protect\rmarkdownfootnote}

%%% Change title format to be more compact
\usepackage{titling}

% Create subtitle command for use in maketitle
\providecommand{\subtitle}[1]{
  \posttitle{
    \begin{center}\large#1\end{center}
    }
}

\setlength{\droptitle}{-2em}

  \title{IMPaC-TB: Integrated analysis and dynamical systems modeling of
experimental TB immunology data}
    \pretitle{\vspace{\droptitle}\centering\huge}
  \posttitle{\par}
    \author{Brooke Anderson PhD, Mike Lyons PhD, Amy Fox, Burton Karger}
    \preauthor{\centering\large\emph}
  \postauthor{\par}
      \predate{\centering\large\emph}
  \postdate{\par}
    \date{2019-12-12}

\usepackage{booktabs}
\usepackage{amsthm}
\makeatletter
\def\thm@space@setup{%
  \thm@preskip=8pt plus 2pt minus 4pt
  \thm@postskip=\thm@preskip
}
\makeatother

\begin{document}
\maketitle

{
\setcounter{tocdepth}{1}
\tableofcontents
}
This is a laboratory guide book for the CSU IMPAC-TB experiments.

\chapter{Introduction}\label{intro}

The overall objective of the data analysis and mathematical modeling
component for the CSU mouse immunology experimental studies is to
develop an iterative framework to identify key biological components of
immunity and to quantify their relationships to one another in both
data-driven and mechanistic models for the purpose of evidence-based
decision-making for tuberculosis (TB) vaccine development. The data
sharing plan (DSP) will include the experimental data, the quantitative
analysis framework, and an application programming interface (API) to
the results. As detailed in our respective biosketches, we have
established expertise in all critical areas of this data analysis
project.

The host immune response to TB vaccination and infection is complex and
involves interactions between large networks of molecular and cellular
constituents that vary in time and location within the host. The
experimental data will be generated from a wide range of measurements
and across multiple scales; including measures of disease pathology,
cellular and chemical measurements for cell type and cytokine
concentrations, and intracellular measurements involving RNA expression
and proteomics. The conceptual basis for our proposed data analysis and
modeling framework is described in a summary of the recent National
Institute of Allergy and Infectious Diseases (NIAID) workshop, `Complex
Systems Science, Modeling and Immunity' {[}1{]}. The major components of
this framework are illustrated in Figure 1, where our approach will
integrate experimental data with data-driven modeling to identify
significant correlations and possible causal structures among the data
elements, and with mechanistic modeling of cell-mediated immunity that
translates biologically-based hypotheses into a dynamical system of
time-dependent mathematical equations that can be used to simulate and
test these hypotheses and to inform the design of subsequent
experiments.

The proposed work plan begins with the collection and organization of
quantitative and qualitative CSU generated experimental data that will
then be used for data-driven and mechanistic modeling, with the analysis
results and software modeling tools being made available through a
web-based API. The milestones of this project are: (1) establishing
protocols and standardized documentation for data collection and
pre-processing from each CSU experimental type, (2) construction of the
relational database (RDB) for CSU-generated experimental data, (3)
collection of qualitative data describing key immune features as input
for mechanistic modeling, (4) development of single-type data-driven
analysis tools for each separate experimental system, (5) development of
integrated data-driven analysis tools for the combined experimental
data, (6) development of a dynamical systems model of cell-mediated
immunity based on qualitative analysis results, (7) development of
parameter estimation and model calibration procedures for the dynamical
systems model, (8) development of software tools that provide for a
start-to-finish process framework, and (9) development of an API for
public access to relevant data and results. For each milestone, the
gates for Go/No Go decisions will be based on the positive
reproducibility of the major results by each of the individual CSU
investigators. This approach will ensure the integration and quality
control of each component within the entire framework.

Immunology data is collected from each CSU mouse experiment as both
quantitative measurements and qualitative data that includes hypotheses
regarding key biological constituents in the context of TB vaccine
development. An RDB will be developed to provide access and queries to
all combined data sets. Data-driven modeling will proceed directly from
the quantitative data while mechanistic modeling will begin with the
qualitative data, with the two modeling approaches increasingly
informing each other as analyses proceed. The data-driven integrated
data analysis will include visualization and statistical analyses and
will also inform parameter estimation for the dynamical systems model.
Software tools will be developed for all quantitative data and results,
including user testing. A tailored user interface will provide access to
all data and analysis results.

Methods. A series of three dedicated network database and computational
data analysis servers will be configured for both internal (CSU) and
remote access to experimental data and to the generated software. The
first in this series will be a low cost, minimal architecture,
development server for use during the base period. This development
server will be then be scaled for intermediate development to establish
resource requirements for an expected maximum workload, followed by a
final production server running the completed workflow applications as
the project deliverable. During the base funding period, we will be
collecting sample datasets from each of the mouse TB infection
experiments to catalog the data and metadata that are being collected.
In subsequent funding periods, we will standardize the (meta)data
collected from these groups and devise a database strategy to store
these research outputs. This will include appropriate scaling for
computer hardware and data storage options. Once the (meta)data is
harmonized, we will construct the relational database. We intend to
leverage open source database software like QPortal {[}5{]} and openBIS
{[}6{]} which are designed to house this type of multi-omics data, to
streamline this process. The final version of the web server will be
housed centrally by Colorado State University's Academic Computing and
Networking Services division and maintained by qualified IT staff. These
systems will also provide a web interface and an API interface that will
allow users to query the data generated by this project, satisfying the
data sharing plan. As data are ready to

be published, they will be deposited to the appropriate NCBI Database
(such as Bioproject, biosample, and SRA) to increase discoverability.
One possible unusual expense would occur in the event that the data
stored locally is lost. In this case, we would incur egress fees from
the cloud storage provider of 1-3 cents per gigabyte of data, depending
on the speed of retrieval. The data-driven modeling will proceed during
the base funding period to develop protocols for computationally
reproducible data collection, pre-processing, analysis, and data-driven
modeling of all data to be collected under the grant. These protocols
will be developed using the rmarkdown {[}7{]} framework, to combine code
with documentation describing all processing, analysis, and data-driven
modeling choices. These protocols will be made publicly available and,
once established, will be used throughout the project to ensure all
experimental data are consistently and reproducibly processed and
analyzed. These protocols will incorporate existing open source software
tools, including xcms {[}8{]}, ramclustR {[}9{]}, flowCore {[}10{]}, and
openCyto {[}11{]}. Further, we will develop our own software tools to
complement these existing tools for the purposes of our research,
including novel tools for visualizing and integrating different types of
data (e.g., metabolomics and flow cytometry). The final result of this
data-driven modeling component will be data type-specific (e.g.,
identification of key metabolites) and integrative across data types
(e.g., quantification of notable associations between specific
metabolites and cell populations).

The mechanistic model component will be developed as a hierarchy of
dynamical systems models of cell mediated immunity to TB infection with
the basic approach described in previous models {[}2,3{]}. During the
base funding period, qualitative data will be used to identify
biologically important factors for the host immunity in each of the
animal models under investigation. Common factors will be represented as
graphs, and quantitative experimental data gaps needed to establish
parameter estimates will be identified. Once appropriate experimental
data is available, parameter estimation will be performed using Bayesian
hierarchical modeling, and model simulations will be conducted using
Monte Carlo methods to account for model uncertainty and population
variability. Anticipated difficulties include additional gaps in
experimental data that limit model identifiability, and hypotheses
regarding the model elements that lead to model predictions that
substantially disagree with corresponding experimental results. These
issues will require additional iterations of model development,
including possible new and targeted experimental studies. Schedule. The
schedule for completion and delivery of items specified in the statement
of work is shown in the table below, where arrows denote the duration
and expected completion date.

References {[}1{]} Vodovotz Y, Xia A, Read EL, Bassaganya-Riera J,
Hafler DA, Sontag E, Wang J, Tsang JS, Day JD, Kleinstein SH, Butte AJ,
Altman MC, Hammond R, Sealfon SC. (2017) Solving Immunology? Trends
Immunol. 38(2):116-127. {[}2{]} Friedman A, Turner J, Szomolay B. (2008)
A model on the influence of age on immunity to infection with
Mycobacterium tuberculosis. Exp Gerontol. 43(4):275-85. {[}3{]}
Wigginton JE, Kirschner D. (2001) A model to predict cell-mediated
immune regulatory mechanisms during human infection with Mycobacterium
tuberculosis. J Immunol. 2001 Feb 1;166(3):1951-67. {[}4{]} Gideon HP,
Skinner JA, Baldwin N, Flynn JL, Lin PL. (2016) Early Whole Blood
Transcriptional Signatures Are Associated with Severity of Lung
Inflammation in Cynomolgus Macaques with Mycobacterium tuberculosis
Infection. {[}5{]} Mohr C, Friedrich A, Wojnar D, Kenar E, Polatkan AC,
et al. (2018) qPortal: A platform for data-driven biomedical research.
PLOS ONE 13(1):
e0191603.\url{https://doi.org/10.1371/journal.pone.0191603} {[}6{]}
Barillari C, Ottoz DSM, Fuentes-Serna JM, Ramakrishnan C, Rinn B, Rudolf
F. (2016) openBIS ELN- LIMS: an open-source database for academic
laboratories. Bioinformatics, 32(4): 638--640,
\url{https://doi.org/10.1093/bioinformatics/btv606} {[}7{]} JJ Allaire,
Yihui Xie, Jonathan McPherson, Javier Luraschi, Kevin Ushey, Aron
Atkins, Hadley Wickham, Joe Cheng and Winston Chang (2018). rmarkdown:
Dynamic Documents for R. R package version 1.9. htt
ps://CRAN.R-project.org/package=rmarkdown {[}8{]} Smith, C.A., Want,
E.J., O'Maille, G., Abagyan,R., Siuzdak, G. (2006). XCMS: Processing
mass spectrometry data for metabolite profiling using nonlinear peak
alignment, matching and identification. Analytical Chemistry, 78,
779--787. {[}9{]} Broeckling, C.D., Afsar, F.A., Neumann, S., Ben-Hur,
A., Prenni, J.E. (2014). RAMClust: a novel feature clustering method
enables spectral-matching-based annotation for metabolomics data.
Analytical Chemistry, 86(14), 6812-6817. {[}10{]} Ellis B, Haaland P,
Hahne F, Le Meur N, Gopalakrishnan N, Spidlen J, Jiang M (2018).
flowCore: flowCore: Basic structures for flow cytometry data. R package
version 1.46.1. {[}11{]} Finak, Greg, Frelinger, Jacob, Jiang, Wenxin,
Newell, Evan W., Ramey, John, Davis, Mark M., Kalams, Spyros A., De
Rosa, Stephen C., Gottardo, Raphael (2014). ``OpenCyto: An Open Source
Infrastructure for Scalable, Robust, Reproducible, and Automated,
End-to-End Flow Cytometry Data Analysis.'' PLoS Computational Biology,
10(8), e1003806.

\chapter{Day -247}\label{day--247}

The purpose of ths timepoint is to identify the cage numbers with the
treatments and create a short cage id.

Note: in future experiments, we will tag the mice at this timepoint.

\href{https://github.com/lyonsm7/impactb_book/raw/master/file_collection_templates/Base_Period/BP_T-247.xlsx}{The
data template files can be found here}

\chapter{Day -242}\label{day--242}

The purpose of this timepoint is to shave the mice before vaccination
and to find the iniital total weight of the mice in the cage before
vaccination.

\href{https://github.com/lyonsm7/impactb_book/raw/master/file_collection_templates/Base_Period/BP_T-242.xlsx}{The
data template files can be found here}

\chapter{Day -240}\label{day--240}

The purpose of this timepoint is to perform the first round of
vaccinations.

\href{https://github.com/lyonsm7/impactb_book/raw/master/file_collection_templates/Base_Period/BP_T-240.xlsx}{The
data template files can be found here}

\chapter{Day -233}\label{day--233}

The purpose of this timepoint is to get the toal cage weight
post-vaccination.

\href{https://github.com/lyonsm7/impactb_book/raw/master/file_collection_templates/Base_Period/BP_T-233.xlsx}{The
data template files can be found here}

\chapter{Deaths}\label{deaths}

Mouse deaths should be recorded in the template.

\href{https://github.com/lyonsm7/impactb_book/raw/master/file_collection_templates/Base_Period/BP_Mouse_Deaths.xlsx}{The
data template files can be found here}

\chapter{Plate Counting}\label{plate-counting}

\href{https://github.com/lyonsm7/impactb_book/raw/master/file_collection_templates/Base_Period/dilution_counts_for_plates.xlsx}{The
dilution count data template files can be found here}

\href{https://github.com/lyonsm7/impactb_book/raw/master/file_collection_templates/Base_Period/inoculum_meta_data_plating.xlsx}{The
counting metadata template files can be found here}

\bibliography{book.bib,packages.bib}


\end{document}
